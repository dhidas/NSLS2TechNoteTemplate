\documentclass{nslsii}
\usepackage{graphicx}

\usepackage[hidelinks]{hyperref}
\usepackage[scale=3]{draftwatermark}

\title{\LaTeX{}  for NSLS II Technical Notes}
\author{The NSLS II Collaboration}
\date{\today}
\notenumber{-1}

\begin{document}

\maketitle

\begin{abstract}
This is an example of how get and to use the nslsii \LaTeX{} class to produce an NSLS II Technical Note.
\end{abstract}



\section{Availability}
You can clone the repository \href{https://github.com/dhidas}{https://github.com/dhidas}:
\begin{verbatim}
> git clone https://github.com/dhidas/asd.git
\end{verbatim}

\section{Usage}
Edit Template.tex, nslsii.cls, and Makefile as you see fit.  Typically to compile the example one only needs to:
\begin{verbatim}
> make
\end{verbatim}
which will generate Template.pdf

\section{Contribute}
Please contribute improvements to the repository for everyone to enjoy.


\section{Figures}
It is nice to throw in a figure such as figure~\ref{fig:example} and some math
$$
\vec E(\vec r, \omega) = \frac{ie\omega}{4 \pi c \epsilon_0} \int_{-\infty}^{+\infty} \frac{\vec \beta - \hat n (1 + \frac{i c}{ \omega D})}{D} e^{i \omega ( \tau + \frac{D}{c} )} d\tau
$$

Some cite~\cite{osti_910923}.

\begin{figure}
\centering
\includegraphics[width=0.5\columnwidth]{img/spectra}
\caption{Example figure caption goes here.}.
\label{fig:example}
\end{figure}


\section*{Acknowledgement}
Thanks mostly to internet search engines and previous \LaTeX{} survivors.




%\bibliographystyle{unsrt}
\bibliographystyle{plain}
\bibliography{\jobname}

\end{document}
